\documentclass{article}
\usepackage{graphicx} % Required for inserting images
\usepackage[polish]{babel}
\usepackage[T1]{fontenc}
\usepackage{amsmath}
\usepackage{minted}
\graphicspath{{./images/}}

\title{Algorytmy macierzowe - Mnożenie macierzy}
\author{Jakub Frączek \and Kacper Garus}

\begin{document}

\sloppy

\maketitle

\tableofcontents

\newpage

\section{Wstęp}

Tematem zadania było wygenerowalnie losowych macierzy o wartościach z przedziału otwartego \((0.00000001, 1.0)\), a następnie zaimplementowanie algorytmów:

\begin{enumerate}
    \item Rekurencyjnego mnożenia macierzy metodą Binét’a
    \item Rekurencyjnego mnożenia macierzy metodą Strassena
    \item Mnożenia macierzy metodą AI na podstawie artykułu w Nature\(^*\)
\end{enumerate}

\noindent
* - https://deepmind.google/discover/blog/discovering-novel-algorithms-with-alphatensor/\#:~:text=In\%20our\%20paper,\%20published\%20today\%20in\%20Nature,\%20we

\bigbreak
\noindent
Następnie zliczyć liczbę operacji zmienno-przecinkowych dokonaną podczas mnożenia macierzy. Algorytmy miały zostać zaprojektowane tak, aby przyjmować macierze o dowolnych wymiarach.

\section{Metoda Binét’a}

\subsection{Opis teoretyczny}

Algrorytm Binét'a jest rekurencyjny i można go przedstawić dla przykładowych macierzy A i B w taki sposób:

\[
A =
\begin{bmatrix}
A_{11} & A_{12} \\
A_{21} & A_{22}
\end{bmatrix}
\quad
B =
\begin{bmatrix}
B_{11} & B_{12} \\
B_{21} & B_{22}
\end{bmatrix}
\]

\[
C =
\begin{bmatrix}
(A_{11}B_{11} + A_{12}B_{21}) & (A_{11}B_{12} + A_{12}B_{22}) \\
(A_{21}B_{11} + A_{22}B_{21}) & (A_{21}B_{12} + A_{22}B_{22})
\end{bmatrix}
\]

\bigbreak
\noindent
Gdzie \(A_{ij}\), \(B_{ij}\) dla \(i = {1, 2, ..., n}\) i \({j = {1, 2, ..., n}}\), to macierze


\subsection{Pseudokod}

\begin{minted}[bgcolor=white]{python}
Funkcja Binet(A, B)
    Jeżeli rozmiar(A) == 1
        Zwróć A * B

    środek = dzielenie_całkowite(rozmiar(A[0]), 2) 
    
    a11 = Wiersze od 0 do środek, Kolumny od 0 do środek z macierzy A
    a12 = Wiersze od 0 do środek, Kolumny od środek do n z macierzy A
    a21 = Wiersze od środek do n, Kolumny od 0 do środek z macierzy A
    a22 = Wiersze od środek do n, Kolumny od środek do n z macierzy A
    
    b11 = Wiersze od 0 do środek, Kolumny od 0 do środek z macierzy B
    b12 = Wiersze od 0 do środek, Kolumny od środek do n z macierzy B
    b21 = Wiersze od środek do n, Kolumny od 0 do środek z macierzy B
    b22 = Wiersze od środek do n, Kolumny od środek do n z macierzy B

    c11 = Binet(a11, b11) + Binet(a12, b21)
    c12 = Binet(a11, b12) + Binet(a12, b22)
    c21 = Binet(a21, b11) + Binet(a22, b21)
    c22 = Binet(a21, b12) + Binet(a22, b22)

    Zwróć macierz C złożóną z c11, c12, c21, c22
\end{minted}

\subsection{Implementacja}

Algorytm postanowiliśmy zaimplementować w języku Python:

\begin{minted}[bgcolor=white]{python}
def binet(a,b):
    
    if np.size(a)==1:
        return a*b
    n=np.size(a[0])
    mid=n//2
    
    a11=a[:mid,:mid]
    a12=a[:mid,mid:]
    a21=a[mid:,:mid]
    a22=a[mid:,mid:]
    b11=b[:mid,:mid]
    b12=b[:mid,mid:]
    b21=b[mid:,:mid]
    b22=b[mid:,mid:]

    c11=binet(a11, b11)+binet(a12, b21)
    c12=binet(a11, b12)+binet(a12, b22)
    c21=binet(a21, b11)+binet(a22, b21)
    c22=binet(a21, b12)+binet(a22, b22)

    return np.vstack((np.hstack((c11,c12)),np.hstack((c21,c22))))
\end{minted}

\section{Metoda Strassena}

\subsection{Opis teoretyczny}

Algorytm Strassena jest rekurencyjny i można go przedstawić dla przykładowych macierzy A i B w następujący sposób:

\[
A =
\begin{bmatrix}
A_{11} & A_{12} \\
A_{21} & A_{22}
\end{bmatrix}
\quad
B =
\begin{bmatrix}
B_{11} & B_{12} \\
B_{21} & B_{22}
\end{bmatrix}
\]

\[
C =
\begin{bmatrix}
P_{1} + P_{4} - P_{5} + P_{7} & P_{2} + P_{4} \\
P_{3} + P_{5} & P_{1} - P_{2} + P_{3} + P_{6}
\end{bmatrix}
\]

\[
\begin{aligned}
P_1 & = (A_{11} + A_{22})(B_{11} + B_{22}) & P_2 & = (A_{21} + A_{22})B_{11} \\
P_3 & = A_{11}(B_{12} - B_{22}) & P_4 & = A_{22}(B_{21} - B_{11}) \\
P_5 & = (A_{11} + A_{12})B_{22} & P_6 & = (A_{21} - A_{11})(B_{11} + B_{12}) \\
P_7 & = (A_{12} - A_{22})(B_{21} + B_{22}) &
\end{aligned}
\]

Gdzie \(A_{ij}\), \(B_{ij}\) dla \(i = {1, 2, ..., n}\) i \({j = {1, 2, ..., n}}\), to macierze

\subsection{Pseudokod}

\begin{minted}[bgcolor=white]{python}
Funkcja Binet(A, B)
    Jeżeli rozmiar(A) == 1
        Zwróć A * B

    środek = dzielenie_całkowite(rozmiar(A[0]), 2) 
    
    a11 = Wiersze od 0 do środek, Kolumny od 0 do środek z macierzy A
    a12 = Wiersze od 0 do środek, Kolumny od środek do n z macierzy A
    a21 = Wiersze od środek do n, Kolumny od 0 do środek z macierzy A
    a22 = Wiersze od środek do n, Kolumny od środek do n z macierzy A
    
    b11 = Wiersze od 0 do środek, Kolumny od 0 do środek z macierzy B
    b12 = Wiersze od 0 do środek, Kolumny od środek do n z macierzy B
    b21 = Wiersze od środek do n, Kolumny od 0 do środek z macierzy B
    b22 = Wiersze od środek do n, Kolumny od środek do n z macierzy B

    p1 = strassen(a11+a22, b11+b22)
    p2 = strassen(a21+a22, b11)
    p3 = strassen(a11, b12-b22)
    p4 = strassen(a22, b21-b11)
    p5 = strassen(a11+a12, b22)
    p6 = strassen(a21-a11, b11+b12)
    p7 = strassen(a12-a22, b21+b22)

    c11=p1+p4-p5+p7
    c12=p3+p5
    c21=p2+p4
    c22=p1-p2+p3+p6

    Zwróć macierz C złożóną z c11, c12, c21, c22
\end{minted}

\subsection{Implementacja}

Algorytm Strassena również został zaimplementowany w języku Python:

\begin{minted}[bgcolor=white]{python}
def strassen(a,b):
    n=np.size(a[0])
    
    if n==1:
        return a*b
    mid=n//2
    
    a11=a[:mid,:mid]
    a12=a[:mid,mid:]
    a21=a[mid:,:mid]
    a22=a[mid:,mid:]
    b11=b[:mid,:mid]
    b12=b[:mid,mid:]
    b21=b[mid:,:mid]
    b22=b[mid:,mid:]

    p1 = strassen(a11+a22, b11+b22)
    p2 = strassen(a21+a22, b11)
    p3 = strassen(a11, b12-b22)
    p4 = strassen(a22, b21-b11)
    p5 = strassen(a11+a12, b22)
    p6 = strassen(a21-a11, b11+b12)
    p7 = strassen(a12-a22, b21+b22)

    c11=p1+p4-p5+p7
    c12=p3+p5
    c21=p2+p4
    c22=p1-p2+p3+p6

    return np.vstack((np.hstack((c11,c12)),np.hstack((c21,c22))))
\end{minted}

\section{Metoda AI}

\subsection{Opis teoretyczny}

Autorzy artykułu "Discovering novel algorithms with AlphaTensor" w Nature postanowili spojrzeć na problem mnożenia macierzy w inny sposób przekształcając go w grę z bardzo dużą liczbą możliwych ruchów, której ukończenie jest równoważne znalezieniu szukanej macierzy. A następnie za pomocą uczenia maszynowego nauczyli model AlphaTensor, jak w nią graćm a ten metodą prób i błędóœ zaczął odkrywać najpierw już znane algorytmy, takie jak metoda Binét’a oraz metoda Strassena, aż dokonał przełomu odkrywając sposób wymnożenia macierzy 4x5 przez macierz 5x5 szybciej niż to było dotychczas możliwe. Najprostszy znany algorytm wykonuje obliczenia przy użyciu 100 mnożeń, algorytm Strassena przy 80, a algorytm AI przy 76.

\begin{figure}[H]
  \centering
    \includegraphics[width=0.8\textwidth]{AI.png}
  \caption{Algorytm wymyślony przez AI}
\end{figure}

\subsection{Implementacjia}

Algorytm wymyśloney przez sztuczną inteligencję również został zaimplementowany w Pythonie:

\begin{minted}[bgcolor=white]{python}
def ai_matrix_mult(a,b):
    a11=a[0,0]
    a12=a[0,1]
    a13=a[0,2]
    a14=a[0,3]
    a15=a[0,4]
    a21=a[1,0]
    a22=a[1,1]
    a23=a[1,2]
    a24=a[1,3]
    a25=a[1,4]
    a31=a[2,0]
    a32=a[2,1]
    a33=a[2,2]
    a34=a[2,3]
    a35=a[2,4]
    a41=a[3,0]
    a42=a[3,1]
    a43=a[3,2]
    a44=a[3,3]
    a45=a[3,4]
    b11=b[0,0]
    b12=b[0,1]
    b13=b[0,2]
    b14=b[0,3]
    b15=b[0,4]
    b21=b[1,0]
    b22=b[1,1]
    b23=b[1,2]
    b24=b[1,3]
    b25=b[1,4]
    b31=b[2,0]
    b32=b[2,1]
    b33=b[2,2]
    b34=b[2,3]
    b35=b[2,4]
    b41=b[3,0]
    b42=b[3,1]
    b43=b[3,2]
    b44=b[3,3]
    b45=b[3,4]
    b51=b[4,0]
    b52=b[4,1]
    b53=b[4,2]
    b54=b[4,3]
    b55=b[4,4]

    h1=a32*(-b21-b25-b31)
    h2=(a22+a25-a35)*(-b25-b51)
    h3=(-a31-a41+a42)*(-b11+b25)
    h4=(a12+a14+a34)*(-b25-b41)
    h5=(a15+a22+a25)*(-b24+b51)
    h6=(-a22-a25-a45)*(b23+b51)
    h7=(-a11+a41-a42)*(b11+b24)
    h8=(a32-a33-a43)*(-b23+b31)
    h9=(-a12-a14+a44)*(b23+b41)
    h10=(a22+a25)*(b51)
    h11=(-a21-a41+a42)*(-b11+b22)
    h12=(a41-a42)*(b11)
    h13=(a12+a14+a24)*(b22+b41)
    h14=(a13-a32+a33)*(b24+b31)
    h15=(-a12-a14)*(b41)
    h16=(-a32+a33)*(b31)
    h17=(a12+a14-a21+a22-a23+a24-a32+a33-a41+a42)*(b22)
    h18=(a21)*(b11+b12+b52)
    h19=(-a23)*(b31+b32+b52)
    h20=(-a15+a21+a23-a25)*(-b11-b12+b14-b52)
    h21=(a21+a23-a25)*(b52)
    h22=(a13-a14-a24)*(b11+b12-b14-b31-b32+b34+b44)
    h23=(a13)*(-b31+b34+b44)
    h24=(a15)*(-b44-b51+b54)
    h25=(-a11)*(b11-b14)
    h26=(-a13+a14+a15)*(b44)
    h27=(a13-a31+a33)*(b11-b14+b15+b35)
    h28=(-a34)*(-b35-b41-b45)
    h29=(a31)*(b11+b15+b35)
    h30=(a31-a33+a34)*(b35)
    h31=(-a14-a15-a34)*(-b44-b51+b54-b55)
    h32=(a21+a41+a44)*(b13-b41-b42-b43)
    h33=(a43)*(-b31-b33)
    h34=(a44)*(-b13+b41+b43)
    h35=(-a45)*(b13+b51+b53)
    h36=(a23-a25-a45)*(b31+b32+b33+b52)
    h37=(-a41-a44+a45)*(b13)
    h38=(-a23-a31+a33-a34)*(b35+b41+b42+b45)
    h39=(-a31-a41-a44+a45)*(b13+b51+b53+b55)
    h40=(-a13+a14+a15-a44)*(-b31-b33+b34+b44)
    h41=(-a11+a41-a45)*(b13+b31+b33-b34+b51+b53-b54)
    h42=(-a21+a25-a35)*(-b11-b12-b15+b41+b42+b45-b52)
    h43=(a24)*(b41+b42)
    h44=(a23+a32-a33)*(b22-b31)
    h45=(-a33+a34-a43)*(b35+b41+b43+b45+b51+b53+b55)
    h46=(-a35)*(-b51-b55)
    h47=(a21-a25-a31+a35)*(b11+b12+b15-b41-b42-b45)
    h48=(-a23+a33)*(b22+b32+b35+b41+b42+b45)
    h49=(-a11-a13+a14+a15-a21-a23+a24+a25)*(-b11-b12+b14)
    h50=(-a14-a24)*(b22-b31-b32+b34-b42+b44)
    h51=(a22)*(b21+b22-b51)
    h52=(a42)*(b11+b21+b23)
    h53=(-a12)*(-b21+b24+b41)
    h54=(a12+a14-a22-a25-a32+a33-a42+a43-a44-a45)*(b23)
    h55=(a14-a44)*(-b23+b31+b33-b34+b43-b44)
    h56=(a11-a15-a41+a45)*(b31+b33-b34+b51+b53-b54)
    h57=(-a31-a41)*(-b13-b15-b25-b51-b53-b55)
    h58=(-a14-a15-a34-a35)*(-b51+b54-b55)
    h59=(-a33+a34-a43+a44)*(b41+b43+b45+b51+b53+b55)
    h60=(a25+a45)*(b23-b31-b32-b33-b52-b53)
    h61=(a14+a34)*(b11-b14+b15-b25-b44+b45-b51+b54-b55)
    h62=(a21+a41)*(b12+b13+b22-b41-b42-b43)
    h63=(-a33-a43)*(-b23-b33-b35-b41-b43-b45)
    h64=(a11-a13-a14+a31-a33-a34)*(b11-b14+b15)
    h65=(-a11+a41)*(-b13+b14+b24-b51-b53+b54)
    h66=(a11-a12+a13-a15-a22-a25-a32+a33-a41+a42)*(b24)
    h67=(a25-a35)*(b11+b12+b15-b25-b41-b42-b45+b52+b55)
    h68=(a11+a13-a14-a15-a41-a43+a44+a45)*(-b31-b33+b34)
    h69=(-a13+a14-a23+a24)*(-b24-b31-b32+b34-b52+b54)
    h70=(a23-a25+a43-a45)*(-b31-b32-b33)
    h71=(-a31+a33-a34+a35-a41+a43-a44+a45)*(-b51-b53-b55)
    h72=(-a21-a24-a41-a44)*(b41+b42+b43)
    h73=(a13-a14-a15+a23-a24-a25)*(b11+b12-b14+b24+b52-b54)
    h74=(a21-a23+a24-a31+a33-a34)*(b41+b42+b45)
    h75=-(a12+a14-a22-a25-a31+a32+a34+a35-a41+a42)*(b25)
    h76=(a13+a33)*(-b11+b14-b15+b24+b34-b35)

    c11=-h10+h12+h14-h15-h16+h53+h5-h66-h7
    c21=h10+h11-h12+h13+h15+h16-h17-h44+h51
    c31=h10-h12+h15+h16-h1+h2+h3-h4+h75
    c41=-h10+h12-h15-h16+h52+h54-h6-h8+h9
    c12=h13+h15+h20+h21-h22+h23+h25-h43+h49+h50
    c22=-h11+h12-h13-h15-h16+h17+h18-h19-h21+h43+h44
    c32=-h16-h19-h21-h28-h29-h38+h42+h44-h47+h48
    c42=h11-h12-h18+h21-h32+h33-h34-h36+h62-h70
    c13=h15+h23+h24+h34-h37+h40-h41+h55-h56-h9
    c23=-h10+h19+h32+h35+h36+h37-h43-h60-h6-h72
    c33=-h16-h28+h33+h37-h39+h45-h46+h63-h71-h8
    c43=h10+h15+h16-h33+h34-h35-h37-h54+h6+h8-h9
    c14=-h10+h12+h14-h16+h23+h24+h25+h26+h5-h66-h7
    c24=h10+h18-h19+h20-h22-h24-h26-h5-h69+h73
    c34=-h14+h16-h23-h26+h27+h29+h31+h46-h58+h76
    c44=h12+h25+h26-h33-h35-h40+h41+h65-h68-h7
    c15=h15+h24+h25+h27-h28+h30+h31-h4+h61+h64
    c25=-h10-h18-h2-h30-h38+h42-h43+h46+h67+h74
    c35=-h10+h12-h15+h28+h29-h2-h30-h3+h46+h4-h75
    c45=-h12-h29+h30-h34+h35+h39+h3-h45+h57+h59

    c=np.array([[c11,c12,c13,c14,c15],[c21,c22,c23,c24,c25],
               [c31,c32,c33,c34,c35],[c41,c42,c43,c44,c45]])
    
    return c
\end{minted}

\section{Porównanie wydajności algorytmów}

\subsection{Zlliczanie liczby operacji zmiennoprzecinkowych}



\subsection{Porównanie czasów działania}

\section{Oszacowanie złożoności obliczeniowej}

Złożoność obliczeniową postanowiliśmy oszacować teoretycznie. Dla algorytmu Binét'a algorytm wykonuje 8 mnożeń na każdym etapie, a liczba podziałów wynosi \(log{2}[n]\), zatem liczba operacji wynosi \(8^{log{2}{n}} = n^3\). Jeśli chodzi o algorytm Strassena wykonuje on 7 mnożeń na każdym etapie, zatem złożoność wynosi \(7^{log{2}{n}} = 2^{2.81}\).

\section{Porównanie wyników z Octave}

\subsection{Octave}

\subsection{Wyniki}

\section{Wnioski}

\end{document}
